\documentclass[unknownkeysallowed]{beamer}
\usepackage[french,english]{babel}
\usepackage{beamer_js}
\usepackage{shortcuts_js}
\usepackage{etex}
\usepackage{csquotes}
\usepackage{fourier}
\nocite{*}
\addbibresource{biblio.bib}

% importer bibilo !!!!!!!!!!!!!! voir projet file attente
\begin{document}


%%%%%%%%%%%%%%%%%%%%%%%%%%%%%%%%%%%%%%%%%%%%%%%%%%%%%%%%%%%%%%%%%%%%%%%%%%%%%%%
%%%%%%%%%%%%%%%%%%%%%%             Headers               %%%%%%%%%%%%%%%%%%%%%%
%%%%%%%%%%%%%%%%%%%%%%%%%%%%%%%%%%%%%%%%%%%%%%%%%%%%%%%%%%%%%%%%%%%%%%%%%%%%%%%



%%%%%%%%%%%%%%%%%%%%%%%%%%%%%%%%%%%%%%%%%%%%%%%%%%%%%%%%%%%%%%%%%%%%%%%%%%%%%%%
\begin{frame}
\bigskip
\bigskip
\begin{center}{
\LARGE\color{marron}
\textbf{HMMA307 : \\ Modèle linéaire généralisé mixte sur jeu de données Estuary}
\textbf{ }\\
\vspace{0.5cm}
}

\color{marron}
\textbf{}
\end{center}

\vspace{0.5cm}

\begin{center}
\textbf{Yani Bouaffad} \\
\vspace{0.1cm}
Université de Montpellier \\
\end{center}
\centering
\includegraphics[width=0.13\textwidth]{Logo}

\end{frame}
%%%%%%%%%%%%%%%%%%%%%%%%%%%%%%%%%%%%%%%%%%%%%%%%%%%%%%%%%%%%%%%%%%%%%%%%%%%%%%%



%%%%%%%%%%%%%%%%%%%%%%%%%%%%%%%%%%%%%%%%%%%%%%%%%%%%%%%%%%%%%%%%%%%%%%%%%%%%%%%
%%%%%%%%%%%%%%%%%%%%%%%%       PLAN      %%%%%%%%%%%%%%%%%%%%%%%%%%%%%%%%%%%%%%
%%%%%%%%%%%%%%%%%%%%%%%%%%%%%%%%%%%%%%%%%%%%%%%%%%%%%%%%%%%%%%%%%%%%%%%%%%%%%%%



%%%%%%%%%%%%%%%%%%%%%%%%%%%%%%%%%%%%%%%%%%%%%%%%%%%%%%%%%%%%%%%%%%%%%%%%%%%%%%%
\begin{frame}{Sommaire}
\tableofcontents[hideallsubsections]
\end{frame}
%%%%%%%%%%%%%%%%%%%%%%%%%%%%%%%%%%%%%%%%%%%%%%%%%%%%%%%%%%%%%%%%%%%%%%%%%%%%%%%


%%%%%%%%%%%%%%%%%%%%%%%%%%%%%%%%%%%%%%%%%%%%%%%%%%%%%%%%%%%%%%%%%%%%%%%%%%%%%%%
%%%%%%%%%%%%%%%%%%%%%%%%%%%%%%%%%%%%%%%%%%%%%%%%%%%%%%%%%%%%%%%%%%%%%%%%%%%%%%%
\section{Introduction}
\label{sec:motiv}
\section{Modèle binaire}
\label{sec:motiv}
\section{Modèle de comptage}
\label{sec:motiv}
\section{Modèle de dénombrement suivant les bryozoaires }
\label{sec:motiv}
\section{Conclusion}
\label{sec:motiv}
%%%%%%%%%%%%%%%%%%%%%%%%%%%%%%%%%%%%%%%%%%%%%%%%%%%%%%%%%%%%%%%%%%%%%%%%%%%%%%%
%%%%%%%%%%%%%%%%%%%%%%%%%%%%%%%%%%%%%%%%%%%%%%%%%%%%%%%%%%%%%%%%%%%%%%%%%%%%%%%
%%%%%%%%%%%%%%%%%%%%%%%%%%%%%%%%%%%%%%%%%%%%%%%%%%%%%%%%%%%%%%%%%%%%%%%%%%%%%%%
\begin{frame}{Introduction}
\begin{itemize}\setlength{\itemsep}{5pt}
\visible<1->{\item \textbf{Données : Estuary}}
\visible<1->{\item \textbf{Modele : GLMM}}
\visible<1->{\item \textbf{Analyse : dénombrement et la présences/ absences d'espèces individuelles}}

\\

\end{itemize}
\end{frame}


\begin{frame}{}

Pour tester s'il y a un effet de modification sur le nombre d'espèces individuelles et la présence / absence,nous devons utiliser des modèles mixtes linéaires généralisés.

\visible<1->{\item \textbf{glmer sur R}}
\end{frame}

\begin{frame}{Hypothèses}
 \visible<1->{\item \textbf{1.}}
 Les observés y sont indépendants, conditionnés par certains prédicteurs x.\\
 \visible<1->{\item \textbf{2.}}Les réponses proviennent d'une distribution connue de la famille exponentielle, avec une relation de variance moyenne connue.\\
 \visible<1->{\item \textbf{3.}} Il existe une relation en ligne droite entre une fonction connue (lien) de la moyenne de y et les prédicteurs x et effets aléatoires z.\\
 \visible<1->{\item \textbf{4.}} Effets aléatoires z sont indépendants de y.\\
 \visible<1->{\item \textbf{5.}} Effets aléatoires z sont normalement distribués.\\
\end{frame}
\begin{frame}{données binaires}



\visible<1->{\item \textbf{1.Tracés résiduels difficiles à interpréter}}

\visible<1->{\item \textbf{2.On ne peut pas trouver de modèle}}

\end{frame}


\begin{frame}{données binaires : Test d'hypothèses}

En examinant brièvement nos hypothèses,nous ne pouvons pas vérifier les hypothèses 1 et 4 , mais elles seront vraies si nous échantillonnons au hasard. Nous devrions vérifier les Hypothèses 2 et 3,  avec les graphiques résiduels, mais étant donné ses défauts, nous ne sommes pas sûrs. L'hypothèse 5 est difficile à vérifier en général et n'est pas cruciale.


\end{frame}


\begin{frame}{Résultat}
    
 \visible<1->{\item \textbf{Bootstrapping}}\\
 Procédure statistique qui rééchantillonne un seul jeu de données pour créer de nombreux échantillons simulés. 
    
    
La p-value obtenue est supérieure à la valeur seuil de 0,05 et ainsi, on ne peut pas rejeter l'hypothèse nulle : qu'il n'y a pas d'effet de la variable modification sur la présence d'hydroïdes.
    

\end{frame}



\begin{frame}{Données de comptage}

 \visible<1->{\item \textbf{glmer avec la loi de poisson.}}\\

Une fois de plus, les graphiques des résidus ne sont pas très utiles, mais nous avons au moins une idée du caractère raisonnable de l’hypothèse de la variance. Il n'y a pas de forme d'éventail évidente, donc un modèle de Poisson semble correct.

 \visible<1->{\item \textbf{Bootsrapping}}\\
 
 La valeur p obtenue est supérieure à la valeur seuil de 0,05 et donc, nous ne pouvons pas rejeter l'hypothèse nulle, à savoir qu'il n'y a pas d'effet de modification sur l'abondance des hydroïdes.
 
 
\end{frame}


\begin{frame}{Dénombrement à partir de Schizoporella errata}

 \visible<1->{\item \textbf{Rien de concluant}}\\


    
\end{frame}


\begin{frame}{Conclusion}

J'ai eu beaucoup de difficulté à pouvoir recoder le code en python surtout à cause de la fonction glmer. Cela m'a pris beaucoup de temps et je regrette la pauvreté de mon beamer.
Cependant j'ai pu à travers ce jeu de données comprendre la démarche de la statiscienne et l'impact écologique qu'elle ne prouve pas mais soulève.
    
\end{frame}
%%%%%%%%%%%%%%%%%%%%%%%%%%%%%%%%%%%%%%%%%%%%%%%%%%%%%%%%%%%%%%%%%%%%%%%%%%%%%%%


\end{document}
